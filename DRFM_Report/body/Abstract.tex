\begin{abstract}
Digital Radio Frequency Memory (DRFM) is a technique used to record an incoming Radio Frequency (RF) signal, in turn applying a series of time-delays, amplitude scalings and frequency shifts and retransmitting the signal. This technique is used widely in the electronic defense industry as a form of radar jamming, in that it allows for the synthesis of artificial targets. This paper discusses the design of implementation of a DRFM system on a low cost FPGA system and documents this system's performance. {\color{red} Describe results of the implementation} sdf sadf sdf asdf asdf asdfsdfasdfadfsd fsdlfjlsdjflsj sdjfl sdjfl jsdlf jsdlf jsdlfj sdljsdlfjsdl jasdfsdjl asdjflsdjl sdjfsdlfj sldfjaslf jdlfjsdlfaljflasjflk jsdlf jasdlfjaslfjsdflsjadlfsjadlf jsdfl jasdlf jsdlkf jasdlf jasdl jasdlasdjl asdjlj lfjadlfjsfjasdflajsflajsfljas lkasdf lsadj lfjsdfl a asdf asdf asd fasf asdf asd asd fasd fasdf asdf asdf asdf asd asd afasd fasdf sdfasf asdfsfsdfsdfsd dfg dfg dfg dfg sr wer sdfl ;'
\end{abstract} 
